\chapter{Conclusion and Future Work}\label{chapter:conclusion}

The main goal of this thesis was to create a metric for quick and objective evaluation of hand posture comfort and discomfort. Furthermore, the metric's relevance for design of hand postures should be demonstrated by proving its influence on precision and performance in a 3D pointing task. 
For the creation of the metrics known influences on comfort and discomfort were combined with newly created components based on state-of-the-art comfort and discomfort models applied to the special context of the hand. Furthermore, data from a user study was used to improve the metrics. Results from the testing user study suggest the improved metric to be a valid extrapolation of the training data. Therefore, the improved metric can be used to predict user perceived comfort and discomfort of a certain hand posture.

However, the improved metric is far from perfect. Due to the small number of participants used for the creation, the computed formula is noisy and has some room for improvement. In addition, comfort and discomfort causing factors were not strictly separated during the creation of the metric. These might have different short- and long-term effects on precision and performance. 

However, the created metric served the purpose of indicating the existence of a correlation between comfort/discomfort and precision/performance, as it was found in the target shooting test results. This correlation was already indicated for other contexts in other papers.

Even though the results of this paper only apply to hand posture comfort/discomfort, the process of creating and testing a comfort/discomfort metric and its influence on performance/precision, as shown in this paper, can be transferred to other parts of the body. In addition to the primary search results, a bunch of byproducts were created during the implementation. Thus, a flexible environment for hand tracking and hand posture classification based on a k-NN algorithm was created that can be used with the Leap or other finger tracking devices to create for example a game using free hand postural user interaction.

Although the results of this thesis give a big first hint about the importance of comfort and discomfort considerations for hand postures, more work has to be done to get a solid model and metric for hand posture comfort and discomfort. For this the models for hand posture comfort and discomfort have to be refined, new influencing factors have to be determined and the computation of the existing components should be optimized. When creating the metric, either comfort and discomfort components should be strictly separated or a metric should be created that preeminently focuses on the performance influences of different hand postures. Another important next step would be to differentiate between short-term and long-term effects of comfort and discomfort in order to enable specific selection of hand postures for different user contexts.