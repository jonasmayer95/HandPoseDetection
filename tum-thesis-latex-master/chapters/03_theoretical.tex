\chapter{Theoretical Foundation}\label{chapter:theory}

\section{Comfort and Discomfort Definitions}

Before actually creating a metric, it is obviously crucial to have an exact definition of the terms comfort and discomfort. In this bachelor thesis comfort and discomfort will be referred to as described by Vink et al. \cite{vink2012editorial}.

In their paper comfort is defined as "pleasant state or relaxed feeling of a human being in reaction to its environment". Therefore comfort is a positive emotional state in reaction to the environment highly dependent on emotions and expectation. Comfort is generally related to "luxury, feeling relaxed or being refreshed".

Discomfort on the other hand is defined as "an unpleasant state of the human body in reaction to its physical environment". Physical stress is the main cause of discomfort, a negative state of the body. Discomfort is often felt in the form of fatigue, stiffness and pain and can in extreme cases even lead to injury.

It is important to keep in mind, that comfort and discomfort are in fact not two opposing sides on one scale. They much more are two independent factors influencing the overall well being in different ways, somewhat similar to Herzberg's motivation-hygiene theory. The absence of discomfort does not automatically result in comfort and vice versa. 

An example for the importance of this differentiation can be found when choosing the softness of foams for mattresses or seats. While softer materials will continuously increase perceived comfort, having too soft foams will result in reduced postural support, leading to higher stress on muscles and tendons and finally causing discomfort symptoms like stiffness or back pain. 


\section{Hand Comfort and Discomfort Metric}

\section{Naive and Improved Metric}
